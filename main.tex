% This template was originally by R. Jacob Vogelstein
% Updated on March 1, 2010 by Noah J. Cowan
% Updated on May 18, 2014 by Brian Weitzner at https://github.com/weitzner/jhu-thesis-template
% Updated on January 29, 2016 by John Muschelli at https://github.com/muschellij2/PhD_Thesis
% Updated on April 13, 2016 by Leonardo Collado Torres and available at https://github.com/lcolladotor/jhu-thesis-template. View (read-only) at Overleaf here https://www.overleaf.com/read/tqdzgmrxgbtg

%% It's your responsability to make sure that your thesis complies with
%% JHU's formatting rules available at http://guides.library.jhu.edu/etd/formatting

\documentclass[12pt]{report}

%% This was the setup recommended at https://github.com/weitzner/jhu-thesis-template
% \documentclass[12pt,oneside,final]{thesis}

\pdfminorversion=4\relax
\pdfobjcompresslevel=0\relax
%% Followed the information from https://www.overleaf.com/latex/examples/creating-pdf-slash-a-and-pdf-slash-x-files/bbbycnbyqhnm#.Vw6_XBMrLm1 to create a PDF/A file in Overleaf
\usepackage[a-1b]{pdfx} % Need this to create a PDF/A file

\usepackage{pdfpages}

\pagestyle{myheadings}
%\topmargin=0.25in
\topmargin=0.05in
\textheight=8.15in
\textwidth=5.6in
\oddsidemargin=0.7in
\raggedbottom
\newdimen \jot \jot=5mm
\brokenpenalty=10000


%\usepackage[utf8]{inputenc} % Seems to cause a conflict with fontenc and lmodern
%\DeclareUnicodeCharacter{00A0}{ }
\usepackage[T1]{fontenc}
\usepackage{lmodern} % load a font with all the characters
%\usepackage{hyperref}
\usepackage{tocbibind} % need this to contents adding for TOC
\usepackage{setspace}
\setstretch{1.05}
\usepackage{RJournal_nogeom} % Changes the colors of links among other things
%\usepackage[all]{hypcap}
\usepackage[hypcap=true]{caption}
\hypersetup{linktocpage}
\usepackage{amsmath,amssymb,array}
\usepackage{booktabs}
\usepackage{subfig}

%% load any required packages here
\usepackage{graphicx}
\usepackage{float}
\usepackage{tikz}
\usepackage{graphics}
\usetikzlibrary{positioning}
\usetikzlibrary{shapes,arrows}
\usepackage{dcolumn}
% A math shortcut frequently used by John Muschelli
\newcommand{\bbeta}{\mbox{\boldmath $\beta$}}

%%%%%%%%%%%%%%%%%%%%%%%%%%%%%%%%%%%%%%%%%%%%%%%%%%%%%%%%%%%%%%%%
% DOI from Segmentation
% Don't use - needs hyperref
%%%%%%%%%%%%%%%%%%%%%%%%%%%%%%%%%%%%%%%%%%%%%%%%%%%%%%%%%%%%%%%%
%\makeatletter
%\providecommand{\doi}[1]{%
%  \begingroup
%    \let\bibinfo\@secondoftwo
%    \urlstyle{rm}%
%    \href{http://dx.doi.org/#1}{%
%      doi:\discretionary{}{}{}%
%      \nolinkurl{#1}%
%    }%
%  \endgroup
%}
%\makeatother

%%%%%%%%%%%%%%%%%%%%%%%%%%%%%%%%%%%%%%%%%%%%%%%%%%%%%%%%%%%%%%%%
% StartKNITR STUFF -- added by John Muschelli
%%%%%%%%%%%%%%%%%%%%%%%%%%%%%%%%%%%%%%%%%%%%%%%%%%%%%%%%%%%%%%%%
\usepackage{color}
%% maxwidth is the original width if it is less than linewidth
%% otherwise use linewidth (to make sure the graphics do not exceed the margin)
\makeatletter
\def\maxwidth{ %
  \ifdim\Gin@nat@width>\linewidth
    \linewidth
  \else
    \Gin@nat@width
  \fi
}
\makeatother

\definecolor{fgcolor}{rgb}{0.345, 0.345, 0.345}
\newcommand{\hlnum}[1]{\textcolor[rgb]{0.686,0.059,0.569}{#1}}%
\newcommand{\hlstr}[1]{\textcolor[rgb]{0.192,0.494,0.8}{#1}}%
\newcommand{\hlcom}[1]{\textcolor[rgb]{0.678,0.584,0.686}{\textit{#1}}}%
\newcommand{\hlopt}[1]{\textcolor[rgb]{0,0,0}{#1}}%
\newcommand{\hlstd}[1]{\textcolor[rgb]{0.345,0.345,0.345}{#1}}%
\newcommand{\hlkwa}[1]{\textcolor[rgb]{0.161,0.373,0.58}{\textbf{#1}}}%
\newcommand{\hlkwb}[1]{\textcolor[rgb]{0.69,0.353,0.396}{#1}}%l
\newcommand{\hlkwc}[1]{\textcolor[rgb]{0.333,0.667,0.333}{#1}}%
\newcommand{\hlkwd}[1]{\textcolor[rgb]{0.737,0.353,0.396}{\textbf{#1}}}%

\usepackage{framed}
\makeatletter
\newenvironment{kframe}{%
 \def\at@end@of@kframe{}%
 \ifinner\ifhmode%
  \def\at@end@of@kframe{\end{minipage}}%
  \begin{minipage}{\columnwidth}%
 \fi\fi%
 \def\FrameCommand##1{\hskip\@totalleftmargin \hskip-\fboxsep
 \colorbox{shadecolor}{##1}\hskip-\fboxsep
     % There is no \\@totalrightmargin, so:
     \hskip-\linewidth \hskip-\@totalleftmargin \hskip\columnwidth}%
 \MakeFramed {\advance\hsize-\width
   \@totalleftmargin\z@ \linewidth\hsize
   \@setminipage}}%
 {\par\unskip\endMakeFramed%
 \at@end@of@kframe}
\makeatother

\definecolor{shadecolor}{rgb}{.97, .97, .97}
\definecolor{messagecolor}{rgb}{0, 0, 0}
\definecolor{warningcolor}{rgb}{1, 0, 1}
\definecolor{errorcolor}{rgb}{1, 0, 0}
\newenvironment{knitrout}{}{} % an empty environment to be redefined in TeX
\makeatletter
\newcommand\gobblepars{%
    \@ifnextchar\par%
        {\expandafter\gobblepars\@gobble}%
        {}}
\makeatother
%%%%%%%%%%%%%%%%%%%%%%%%%%%%%%%%%%%%%%%%%%%%%%%%%%%%%%%%%%%%%%%%
% End KNITR STUFF
%%%%%%%%%%%%%%%%%%%%%%%%%%%%%%%%%%%%%%%%%%%%%%%%%%%%%%%%%%%%%%%%


\usepackage[
style = ieee, % follows http://ieeeauthorcenter.ieee.org/wp-content/uploads/IEEE-Reference-Guide.pdf
sorting = none,
dashed = false,
maxbibnames = 99,
backend = bibtex,
natbib = true
]{biblatex}

% If you want to exclude some portions from the bibliography
\AtEveryBibitem{
\clearfield{note}
\clearfield{month}
}


\usepackage{enumerate}

%\tolerance=10000

%\makeglossary % enable the glossary
\graphicspath{{rnw_chapter/figure/}{rnw_chapter/}} % change it accordingly!


\setcounter{tocdepth}{4}
\setcounter{secnumdepth}{4}
\begin{document}

\newcommand{\bm}[1]{ \mbox{\boldmath $ #1 $} }
\newcommand{\bin}[2]{\left(\begin{array}{@{}c@{}} #1 \\ #2
             \end{array}\right) }
\renewcommand{\contentsname}{Table of Contents}
\baselineskip=24pt

% Create cover page of dissertation !
\pagenumbering{roman}
\thispagestyle{empty}
\begin{center}
\vspace*{.25in}
% #TODO: is this title still good?
{\bf\LARGE{ FPGA-Based Adaptive Digital Beamforming Using\\
Machine Learning for MIMO Systems }}\\
\vspace*{.75in}
{\bf by} \\*[18pt]
\vspace*{.2in}
{\bf John Gentile}\\
\vspace*{1in}
{\bf A dissertation submitted to The Johns Hopkins University\\
in conformity with the requirements for the degree of\\
Master of Science }\\
\vspace*{.75in}
{\bf Baltimore, Maryland} \\
{\bf May, 2021} \\
\vspace*{.5in}
\begin{small}
{\bf \copyright{ }2021 by John Gentile} \\
{\bf All rights reserved}
\end{small}
\end{center}
\newpage

% Add acknowledgements
\pagestyle{plain}
\pagenumbering{roman}
\setcounter{page}{2}
\chapter*{Abstract}

My thesis is about bla bla
 % #TODO: update!
\chapter*{Thesis Committee}

\section*{}
\subsection*{Primary Readers}

\begin{singlespace}


\indent Jeff Houser(Primary Advisor)\\
\indent \indent Johns Hopkins University Whiting School of Engineering\\
\indent \indent  Engineering for Professionals\\


\smallskip

\noindent Ashutosh Dutta\\
\indent \indent Johns Hopkins University Whiting School of Engineering\\
\indent \indent  Engineering for Professionals\\

\smallskip

\noindent Doug Wendstrand\\
\indent \indent Johns Hopkins University Whiting School of Engineering\\
\indent \indent  Engineering for Professionals\\

\smallskip

\noindent Ramsey Hourani\\
\indent \indent Johns Hopkins University Whiting School of Engineering\\
\indent \indent  Engineering for Professionals\\

\end{singlespace}


\chapter*{Acknowledgments}

Thanks! % #TODO: update!

%\cleardoublepage
%\newpage
\pagestyle{plain}
\baselineskip=24pt
\tableofcontents
% for the three lines below, change the page numbers if needed!
%\addtocontents{toc}{\contentsline{chapter}{Table of Contents}{iii}}
%\addtocontents{toc}{\protect\contentsline{chapter}{\protect\numberline{}Table of Contents}{iii}}
%\addtocontents{toc}{\protect\contentsline{chapter}{\protect\numberline{}List of Tables}{iv}}
%\addtocontents{toc}{\protect\contentsline{chapter}{\protect\numberline{}List of Figures}{v}}
\listoftables
\listoffigures

\cleardoublepage % Needed because our intro chapter doesn't really have anything
\pagenumbering{arabic}


% add your chapters, best way is to have separate TeX files for each chapter
%\include{chapter0}
%\include{chapter1}

%% The above was the recommended setup by https://github.com/weitzner/jhu-thesis-template but it's no longer needed
%% after Muschelli's changes which stores different chapters in their
%% respective directories. You will still need to add your chapters as
%% TeX files or Rnw files (see rnw_chapter as an example) and please
%% remember to update the makefile accordingly.

\begin{refsection}[intro_chapter/intro_chapter.bib]
\chapter{Introduction}
\label{chap:intro}

Introduce your thesis \citep{A}

\section{Stuff}

ASFGGG asdf cat in the hat

\cleardoublepage
\printbibliography[title={References}]
\end{refsection}

%% If using Overleaf, you'll have to upload the resulting tex file & figures
%% since Overleaf does not support Rnw files right now (April 13, 2016)
\begin{refsection}[rnw_chapter/rnw_chapter.bib]
\include{rnw_chapter/rnw_chapter}
\cleardoublepage
\printbibliography[title={References}]
\end{refsection}

\begin{refsection}[conclusion_chapter/conclusion_chapter.bib]
\chapter{Discussion and Conclusion}
\label{chap:conclusion}

Discuss and conclude your thesis \citep{C}

\cleardoublepage
\printbibliography[title={References}]
\end{refsection}


\end{document}
