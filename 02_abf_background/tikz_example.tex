%% TODO below is scratchpad for tikz stuff & tables

% example table
\begin{table}[ht]
\centering
\begin{tabular}{rr}
  \hline
A & B \\
  \hline
  1 & -0.40 \\
    2 & -1.27 \\
    3 & -1.04 \\
    4 & 0.69 \\
    5 & 0.08 \\
    6 & 1.71 \\
    7 & 1.90 \\
    8 & 1.88 \\
    9 & 0.17 \\
   10 & -1.62 \\
   \hline
\end{tabular}
\caption{\bf{ Table title } Table description}
\label{tab:rtab1}
\end{table}


\tikzset{
block/.style = {draw, fill=white, rectangle, minimum height=3em, minimum width=3em},
tmp/.style  = {coordinate},
sum/.style= {draw, fill=white, circle, node distance=1cm},
input/.style = {coordinate},
output/.style= {coordinate},
pinstyle/.style = {pin edge={to-,thin,black}
}
}

% htbp to ease floating placement https://tex.stackexchange.com/questions/2275/keeping-tables-figures-close-to-where-they-are-mentioned
\begin{figure}[!htbp]
\centering
\begin{tikzpicture}[auto, node distance=2cm,>=latex']
    \node [input, name=rinput] (rinput) {};
    \node [sum, right of=rinput] (sum1) {};
    \node [block, right of=sum1] (controller) {$k_{p\beta}$};
    \node [block, above of=controller,node distance=1.3cm] (up){$\frac{k_{i\beta}}{s}$};
    \node [block, below of=controller,node distance=1.3cm] (rate) {$sk_{d\beta}$};
    \node [sum, right of=controller,node distance=2cm] (sum2) {};
    \node [block, above = 2cm of sum2](extra){$\frac{1}{\alpha_{\beta2}}$};  %
    \node [block, right of=sum2,node distance=2cm] (system)
{$\frac{a_{\beta 2}}{s+a_{\beta 1}}$};
    \node [output, right of=system, node distance=2cm] (output) {};
    \node [tmp, below of=controller] (tmp1){$H(s)$};
    \draw [->] (rinput) -- node{$R(s)$} (sum1);
    \draw [->] (sum1) --node[name=z,anchor=north]{$E(s)$} (controller);
    \draw [->] (controller) -- (sum2);
    \draw [->] (sum2) -- node{$U(s)$} (system);
    \draw [->] (system) -- node [name=y] {$Y(s)$}(output);
    \draw [->] (z) |- (rate);
    \draw [->] (rate) -| (sum2);
    \draw [->] (z) |- (up);
    \draw [->] (up) -| (sum2);
    \draw [->] (y) |- (tmp1)-| node[pos=0.99] {$-$} (sum1);
    \draw [->] (extra)--(sum2);
    \draw [->] ($(0,1.5cm)+(extra)$)node[above]{$d_{\beta 2}$} -- (extra);
    \end{tikzpicture}
    \caption{A PID Control System}
    \label{fig:pid}
\end{figure}



% reuse from Kalman example https://texample.net/tikz/examples/kalman-filter/
\begin{figure}[htbp]
\centering
% The state vector is represented by a blue circle.
% "minimum size" makes sure all circles have the same size
% independently of their contents.
\tikzstyle{state}=[circle,
                   thick,
                   minimum size=1.2cm,
                   draw=blue!80,
                   fill=blue!20]

% The measurement vector is represented by an orange circle.
\tikzstyle{measurement}=[circle,
                         thick,
                         minimum size=1.2cm,
                         draw=orange!80,
                         fill=orange!25]

% The control input vector is represented by a purple circle.
\tikzstyle{input}=[circle,
                   thick,
                   minimum size=1.2cm,
                   draw=purple!80,
                   fill=purple!20]

% The input, state transition, and measurement matrices
% are represented by gray squares.
% They have a smaller minimal size for aesthetic reasons.
\tikzstyle{matrx}=[rectangle,
                   thick,
                   minimum size=1cm,
                   draw=gray!80,
                   fill=gray!20]

% The system and measurement noise are represented by yellow
% circles with a "noisy" uneven circumference.
% This requires the TikZ library "decorations.pathmorphing".
\tikzstyle{noise}=[circle,
                   thick,
                   minimum size=1.2cm,
                   draw=yellow!85!black,
                   fill=yellow!40,
                   decorate,
                   decoration={random steps,
                               segment length=2pt,
                               amplitude=2pt}]

% Everything is drawn on underlying gray rectangles with
% rounded corners.
\tikzstyle{background}=[rectangle,
                        fill=gray!10,
                        inner sep=0.2cm,
                        rounded corners=5mm]

\begin{tikzpicture}[>=latex,text height=1.5ex,text depth=0.25ex]
  % "text height" and "text depth" are required to vertically
  % align the labels with and without indices.

  % The various elements are conveniently placed using a matrix:
  \matrix[row sep=0.5cm,column sep=0.5cm] {
    % First line: input sample array
        \node (other_in0)  {$\vdots$}; &
        \node (other_in1)  {$\vdots$}; &
        \node (other_in2)  {$\vdots$}; &
        \node (other_in3)  {$\vdots$}; &
        \\
        \node (t2_in0)  {$x(2)$}; &
        \node (t2_in1)  {$x(1)$}; &
        \node (t2_in2)  {$x(0)$}; &
        \node (t2_in3)  {$d(2)$}; &
        \\
        \node (t1_in0)  {$x(1)$}; &
        \node (t1_in1)  {$x(0)$}; &
        \node (t1_in2)  {$0$};    &
        \node (t1_in3)  {$d(1)$}; &
        \\
        \node (t0_in0)  {$x(0)$}; &
        \node (t0_in1)  {$0$};    &
        \node (t0_in2)  {$0$};    &
        \node (t0_in3)  {$d(0)$}; &
        \\
    % First line: Control input
    &
        \node (u_k-1) [input]{$\mathbf{u}_{k-1}$}; &
        &
        \node (u_k)   [input]{$\mathbf{u}_k$};     &
        &
        \node (u_k+1) [input]{$\mathbf{u}_{k+1}$}; &
        \\
        % Second line: System noise & input matrix
        \node (w_k-1) [noise] {$\mathbf{w}_{k-1}$}; &
        \node (B_k-1) [matrx] {$\mathbf{B}$};       &
        \node (w_k)   [noise] {$\mathbf{w}_k$};     &
        \node (B_k)   [matrx] {$\mathbf{B}$};       &
        \node (w_k+1) [noise] {$\mathbf{w}_{k+1}$}; &
        \node (B_k+1) [matrx] {$\mathbf{B}$};       &
        \\
        % Third line: State & state transition matrix
        \node (A_k-2)         {$\cdots$};           &
        \node (x_k-1) [state] {$\mathbf{x}_{k-1}$}; &
        \node (A_k-1) [matrx] {$\mathbf{A}$};       &
        \node (x_k)   [state] {$\mathbf{x}_k$};     &
        \node (A_k)   [matrx] {$\mathbf{A}$};       &
        \node (x_k+1) [state] {$\mathbf{x}_{k+1}$}; &
        \node (A_k+1)         {$\cdots$};           \\
        % Fourth line: Measurement noise & measurement matrix
        \node (v_k-1) [noise] {$\mathbf{v}_{k-1}$}; &
        \node (H_k-1) [matrx] {$\mathbf{H}$};       &
        \node (v_k)   [noise] {$\mathbf{v}_k$};     &
        \node (H_k)   [matrx] {$\mathbf{H}$};       &
        \node (v_k+1) [noise] {$\mathbf{v}_{k+1}$}; &
        \node (H_k+1) [matrx] {$\mathbf{H}$};       &
        \\
        % Fifth line: Measurement
        &
        \node (z_k-1) [measurement] {$\mathbf{z}_{k-1}$}; &
        &
        \node (z_k)   [measurement] {$\mathbf{z}_k$};     &
        &
        \node (z_k+1) [measurement] {$\mathbf{z}_{k+1}$}; &
        \\
    };

    % The diagram elements are now connected through arrows:
    \path[->]
        (A_k-2) edge[thick] (x_k-1)	% The main path between the
        (x_k-1) edge[thick] (A_k-1)	% states via the state
        (A_k-1) edge[thick] (x_k)		% transition matrices is
        (x_k)   edge[thick] (A_k)		% accentuated.
        (A_k)   edge[thick] (x_k+1)	% x -> A -> x -> A -> ...
        (x_k+1) edge[thick] (A_k+1)

        (x_k-1) edge (H_k-1)				% Output path x -> H -> z
        (H_k-1) edge (z_k-1)
        (x_k)   edge (H_k)
        (H_k)   edge (z_k)
        (x_k+1) edge (H_k+1)
        (H_k+1) edge (z_k+1)

        (v_k-1) edge (z_k-1)				% Output noise v -> z
        (v_k)   edge (z_k)
        (v_k+1) edge (z_k+1)

        (w_k-1) edge (x_k-1)				% System noise w -> x
        (w_k)   edge (x_k)
        (w_k+1) edge (x_k+1)

        (u_k-1) edge (B_k-1)				% Input path u -> B -> x
        (B_k-1) edge (x_k-1)
        (u_k)   edge (B_k)
        (B_k)   edge (x_k)
        (u_k+1) edge (B_k+1)
        (B_k+1) edge (x_k+1)
        ;

    % Now that the diagram has been drawn, background rectangles
    % can be fitted to its elements. This requires the TikZ
    % libraries "fit" and "background".
    % Control input and measurement are labeled. These labels have
    % not been translated to English as "Measurement" instead of
    % "Messung" would not look good due to it being too long a word.
    \begin{pgfonlayer}{background}
        \node [background,
                    fit=(w_k-1) (v_k-1) (A_k+1)] {};
        \node [background,
                    fit=(z_k-1) (z_k+1),
                    label=left:Measure:] {};
    \end{pgfonlayer}
\end{tikzpicture}

\caption{Kalman filter system model}
\end{figure}
